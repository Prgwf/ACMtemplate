\section{Tool}
\subsection{Fast IO}
\lstinputlisting{"./codes/tool/fastio.cpp"}

\subsection{日期函数}
\lstinputlisting{"./codes/tool/date.cpp"}

\subsection{Bit compression}
\lstinputlisting{"./codes/tool/bitCompression.cpp"}

\subsection{Random}
\lstinputlisting{"./codes/tool/random.cpp"}

\subsection{Hash map}
\lstinputlisting{"./codes/tool/hashMap.cpp"}

\subsection{树链剖分}
\lstinputlisting{"./codes/tool/HeavyLight.cpp"}

\subsection{单调栈}
\lstinputlisting{"./codes/tool/MonoStack.cpp"}

\subsection{单调队列}
\lstinputlisting{"./codes/tool/MonoDeque.cpp"}

%\subsection{120 bit integer}
%\lstinputlisting{"./codes/tool/integer120.cpp"}

\subsection{Big-integer}
\lstinputlisting{"./codes/tool/BIGN.cpp"}

%\subsection{Bash script}
%\lstinputlisting[language=sh]{"./codes/tool/bashScript.sh"}

\subsection{Code::blocks settings}
	\begin{lstlisting}
gnome-terminal -t $TITLE -x
	\end{lstlisting}
	
\subsection{Bit operation}
	\subsubsection{基本操作}
	注意括号\\
		\begin{table}[htbp]
		\centering
		\begin{tabular}{l|l|l}
		\toprule
		功能 & 示例 & 位运算\\
		\midrule
		返回lsb之后的0的个数 & $(1100010)\rightarrow 1D$ & \_\_builtin\_ctz(x)[x==0时UB]\\
		统计二进制$1$的个数  & $(1100110\rightarrow 4D)$        &\_\_builtin\_popcount(x)\\
		取最后一个1的pos+1(ffs) & $(1000010\rightarrow 2D)$       &\_\_builtin\_ffs(x)\\
		取最后一个1的mask(lsb)   & $(1000010\rightarrow 10)$      & (x \& (-x))\\
		去掉最后一位		  & $(101101\rightarrow 10110)$		   & x shr 1\\
		在最后加一个$0$		 & $(101101\rightarrow1011010)$		 & x shl 1\\
		在最后加一个$1$		 & $(101101\rightarrow1011011)$		 & x shl 1+1\\
		把最后一位变成$1$	   & $(101100\rightarrow101101)$		  & x or 1\\
		把最后一位变成$0$	   & $(101101\rightarrow101100)$		  & x or 1-1\\
		最后一位取反		  & $(101101\rightarrow101100)$		  & x xor 1\\
		把右数第$k$位变成$1$	  & $(101001\rightarrow101101,k=3)$	  & x or (1 shl (k-1))\\
		把右数第$k$位变成$0$	  & $(101101\rightarrow101001,k=3)$	  & x and not (1 shl (k-1))\\
		右数第$k$位取反		 & $(101001\rightarrow101101,k=3)$	  & x xor (1 shl (k-1))\\
		取末三位			  & $(1101101\rightarrow101)$			& x and 7\\
		取末$k$位			   & $(1101101\rightarrow1101,k=5)$	   & x and (1 shl k-1)\\
		取右数第$k$位		   & $(1101101\rightarrow1,k=4)$		  & x shr (k-1) and 1\\
		把末$k$位变成$1$		  & $(101001\rightarrow101111,k=4)$	  & x or (1 shl k-1)\\
		末$k$位取反			 & $(101001\rightarrow100110,k=4)$	  & x xor (1 shl k-1)\\
		把右边连续的$1$变成$0$	& $(100101111\rightarrow100100000)$	& x and (x+1)\\
		把右起第一个$0$变成$1$	& $(100101111\rightarrow100111111)$	& x or (x+1)\\
		把右边连续的$0$变成$1$	& $(11011000\rightarrow11011111)$	  & x or (x-1)\\
		取右边连续的$1$		 & $(100101111\rightarrow1111)$		 & (x xor (x+1)) shr 1\\
		去掉右起第一个$1$的左边 & $(100101000\rightarrow1000)$		 & x and (x xor (x-1))\\
		\bottomrule
		\end{tabular}
	\end{table}
	
	\subsubsection{枚举长为$n$含$k$个$1$的$01$串}
	\begin{lstlisting}[language=c++]
int n = 5,k = 3;
for (int s = (1 << k)-1,u = 1 << n; s < u;) {
	for (int i = 0;i < n;i++)
		printf("%d",(((s>>(n-1-i))&1) == 1));
	printf("\n");
	
	int b = s & -s;
	s = (s+b)|(((s^(s+b))>>2)/b);
}
	\end{lstlisting}

\subsection{Profile of Vim}
\lstinputlisting{"./codes/tool/vimrc"}
