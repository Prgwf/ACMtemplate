\section{Graph}
\subsection{Dinic}
\lstinputlisting{"./codes/graph/Dinic.cpp"}
\subsection{ISAP}
\lstinputlisting{"./codes/graph/ISAP.cpp"}
\subsection{Minimal cost maximal flow}
\lstinputlisting{"./codes/graph/MCMF.cpp"}
\subsection{Johnson Minimal cost flow}
\lstinputlisting{"./codes/graph/minimum_cost_flow.cpp"}
\subsection{Bi-connect}
\lstinputlisting{"./codes/graph/bc.cpp"}
\subsection{Cut and bridge}
\lstinputlisting{"./codes/graph/cutAndBridge.cpp"}
\subsection{Stoer-Wagner}
\lstinputlisting{"./codes/graph/sw.cpp"}
\subsection{Euler path}
\lstinputlisting{"./codes/graph/euler.cpp"}
\subsection{Strongly connected component}
\lstinputlisting{"./codes/graph/sc.cpp"}

\subsection{Match}
\subsubsection{Bipartite graph}
\lstinputlisting{"./codes/graph/match.cpp"}
\subsubsection{Edmonds}
\lstinputlisting{"./codes/graph/edmonds.cpp"}
\subsubsection{KM}
\lstinputlisting{"./codes/graph/km.cpp"}

\subsection{Clique}
\lstinputlisting{"./codes/graph/clique.cpp"}

最大团的压位做法 by Claris \\

\lstinputlisting{"./codes/graph/clique2.cpp"}

\subsection{Spanning tree}
\subsubsection{Count the number of spanning tree}
\lstinputlisting{"./codes/graph/laplacian.cpp"}
\subsubsection{Spanning tree on directed graph}
\lstinputlisting{"./codes/graph/zhuliu.cpp"}

\subsection{Kth shortest path}
\lstinputlisting{"./codes/graph/kthpath.cpp"}
\subsection{LCA}
\lstinputlisting{"./codes/graph/lca.cpp"}
\subsection{Virtual-tree}
\lstinputlisting{"./codes/graph/virtual-tree.cpp"}
\subsection{Stable marriage problem}
	假定有$n$个男生和$m$个女生,理想的拍拖状态就是对于每对情侣$(a,b)$,找不到另一对情侣$(c,d)$使得$c$更喜欢$b$,$b$也更喜欢$c$,同理,对$a$来说也没有$(e,f)$使得$a$更喜欢$e$而$e$更喜欢$a$,当然最后会有一些人落单。这样子一个状态可以称为理想拍拖状态,它也有一个专业的名词叫稳定婚姻。\\
	求解这个问题可以用一个专有的算法,延迟认可算法,其核心就是让每个男生按自己喜欢的顺序逐个向女生表白,例如leokan向一个女生求爱,这个过程中,若这个女生没有男朋友,那么这个女生就暂时成为leokan的女朋友,或这个女生喜欢她现有男朋友的程度没有喜欢leokan高,这个女生也暂时成为leokan的女朋友,而她原有的男朋友则再将就找下一个次喜欢的女生来当女朋友。
\lstinputlisting{"./codes/graph/stableMarriage.cpp"}

